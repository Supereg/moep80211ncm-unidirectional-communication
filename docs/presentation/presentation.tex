\documentclass[11pt]{beamer}

\usepackage[utf8]{inputenc}

\logo{\includegraphics[scale=.5]{TUM_Logo.eps}}
\title[Short version]{Unidirectional NCM}
\subtitle[short version]{Implementation}
\date[2021]{Network Coding}
\author[Andreas Bauer, Lion Steger]{Andreas Bauer, Lion Steger}
\institute[Technische Universität München]{Technische Universität München}

\begin{document}
	\frame{\maketitle}
	
	\begin{frame}{Motivation}
		\begin{itemize}
			\item if two hosts discover each other as neighbors, a session is created
			\item sessions are always bidirectional
			\item coding matrices are split for the two directions
			\item this might be overly complicated?
		\end{itemize}
	\end{frame}

	\begin{frame}{Timeline}
		\begin{itemize}
			\item Understanding the code
			\item Writing a "simulator" which does encoding/decoding on one device
			\item Rewriting the code to our needs
			\item Leaving the simulator behind in favor of unit tests
			\item Integrating the new session/generation functionality into the existing code
		\end{itemize}
	\end{frame}

	\begin{frame}{Our (Non-)Changes}
		Implementing this seems simple... right?
		\begin{itemize}
			\item Use the full coding matrix $\rightarrow$ works even without changes to librlnc!
			\item rewrite most of the calls to the session and generation API:
			\begin{itemize}
				\item acknowledgment and re-transmission scheme
				\item generation window advancement
				\item packet statistics
				\item header structures
				\item utility functions (e.g. remaining space)
			\end{itemize}
		\end{itemize}
	\end{frame}

	\begin{frame}{Our Acknowledgment/Advancement Scheme}
		\begin{itemize}
			\item each encoded packet triggers an acknowledgment
			\item acknowledgments transport the lowest sequence number of the window and the receiver dimension
			\item if the generation window is out of sync for receiver and sender, it is fixed by advancing to the next generation or discarding it
		\end{itemize}
	\end{frame}

	\begin{frame}{Difficulties}
		\begin{itemize}
			\item Unit-Tests with timers
			\item Understanding the code
		\end{itemize}
	\end{frame}
\end{document}