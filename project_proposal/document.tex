\documentclass[a4paper, 11pt]{article}
\usepackage[utf8]{inputenc}

\newcommand{\ilc}[1]{\texttt{#1}} % inline-code

\title{Proposal: Unidirectional NCM}
\author{Andreas Bauer, Lion Steger}
\date{January 2021}

\begin{document}

    \maketitle

    \section{Objective}\label{sec:objective}
    As of now, the NCM module features bidirectional sessions only.
    Our goal is to replace the current session handling with unidirectional session handling.
    \\

    For this change we need to consider the following components: the underlying random linear network coding library (\ilc{rlnc}),
    session management (\ilc{session.c}) and generation management (\ilc{generation.c}).
    \\

    The \ilc{rlnc} library can presumably be used as is, by simply ignoring the reverse direction of the coding matrix.

    Session and generation management on the other hand is to be reengineered from scratch.
    The rewrite will include designing a revised acknowledgment scheme to fit the unidirectional coding scheme.

    \section{Milestones}\label{sec:milestones}
    \begin{enumerate}
        \item Abstract formulation of encoding and acknowledgment scheme, drafting the changes to be made.
        \item Simulator: A designated executable, based on the \ilc{rlnc} library, which decouples our
            research from the ncm modules, specifically not requiring network connectivity.

            This allows us to have minimal overhead when researching and prototyping our implementation.
            Additionally it serves well as a starting point for creating a deep understanding of the \ilc{rlnc} library.

            Ideally, we can prototype all three operations within this decoupled project:
            \ilc{encoding}, \ilc{decoding} and \ilc{recoding}/\ilc{forwarding}.
            As ordered, they represent sub-milestones.
        \item First prototype: Integrate our developments of the simulator into the ncm module,
            replacing the current bidirectional implementation.

            Required functionality: Supports transmission of an ICMP Echo Request/Response.
        \item Optional: Reengineer the rlnc instead of ignoring the reverse direction of the coding matrix.\\
        \textit{Note: While working on the project, it was discovered, that our initial understanding of the rlnc
        library was off and actually no changes were required.}
        \item Final version supporting full functionality as the current bidirectional version.
        \item Final presentation.
    \end{enumerate}

    \section{Timeline}\label{sec:timeline}
    The following timeline highlights steps we want to have \textbf{completed} at the listed date.

    \begin{itemize}
        \item Beginning of February: Understanding the current code base using bidirectional sessions and refresh understanding of linear coding schemes.
        \item Mid of February: Milestone 1.
        \item End of February -- Beginning of March: Have a standalone simulator as described in Milestone 2.
        \item Mid of March: Integration into the ncm module (Milestone 3.).
        \item March 24th: Aim to have the final version ready a week early,
            leaving time for last minute adjustments. (Milestone 5.).
        \item March 31st: Final deadline.
        \item Beginning of April: Prepare the final presentation (Milestone 6.)
    \end{itemize}

\end{document}
